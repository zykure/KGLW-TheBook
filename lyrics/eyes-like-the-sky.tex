\album{Eyes Like The Sky}

\artwork{eyes-like-the-sky.jpg}
\released{2013}{02}{22}
\label{album:eyes-like-the-sky}

%----------------------------------------------------------------------

\song{Eyes Like The Sky}

\writtenby{Mackenzie/Smith}

\note{[All songs narrated by Broderick Smith.]}

The bad, white men call him the \word{devil}. The \word{Yavapai} call him ``Eyes Like the \word{Sky}.'' \\

This story takes place in the hinterlands of the newly formed United States and territories in the years before and after the great conflagration called the Civil War. Men roam and fight each other simply to stay breathing. Muskets give way to repeating rifles, cannons give way to Gatling guns. War nurtures weapons, weapons clear the land. \\

In the \word[desert]{deserts} of the southwest, old hatreds grow into new ones. Old beliefs are shattered by gunfire and charging horses. Into this cauldron of fire rides a young man who becomes a shadowed legend. His name even takes on the mantle of the bogeyman in some homes. \\

Among the first Americans, his name is exalted from wickiups to longhouses from teepees to cliff dwellings. Among the men of the badlands, he is feared for his silent walk and swift, economical dispatching of his enemies. \\

%----------------------------------------------------------------------

\song{Year Of Our Lord}

\writtenby{Mackenzie/Smith}

\note{Chapter 1:}

They'd been watching the farmhouse for a while, maybe all day. There were two of them, both young and fit and \word{desert}-hardened. They were lords of where they lived and they had no foreign teachings from white men or Mexicans. \\

Except for one thing. They hated the Mexicans more than the white men because of their cruelty. and they'd learned cruelty from the Mexicans very well in return. \\

The smoke from the mud house curled up into the \word{sky} like an albino \word{snake}. The two young men watched and counted the white men down in the farmyard. A tall man and two shorter ones, maybe his sons. A woman would occasionally come out from the shack. Get water from the well in the front yard and carry it back inside. A small child would be with her. \\

The men watching on the rim had no calendars so they didn't know the date. 12th of June, year of our Lord, 1854. \\

But one thing they did know: about an hour away were the rest of their party. Eight men, all armed, running smoothly and trackless over the rocks. One of the watchers moved away to tell the main party of what they had seen. \\

The raid was about to start. \\

%----------------------------------------------------------------------

\song{The Raid}

\writtenby{Mackenzie/Smith}

They were not after the money, they were not after alcohol. They were after guns and young children to raise as their own. War had made it necessary to take child captives. The rest would be slaughtered. \\

And that is how \word{Miguel O'Brien} became a \word{Yavapai}-Apache warrior.
He was five years old. \\

%----------------------------------------------------------------------

\song{Drum Run}

\writtenby{Mackenzie/Smith}

\note{Chapter 2:}

\word{Miguel O'Brien} ran with the Apaches. He ran and ran, and as his legs grew he glided over the \word{desert} earth. He learned how to hide and to hunt. He learned to leave no tracks and he learned to live on what he could keep down. And his name was now ``Eyes Like The \word{Sky}''. \\

His blue eyes showed his father's race. He never wore the white painted face of the slave. He was valued for his stamina and distant vision. By the time he was 15 he had already killed Mexican troopers and feared no man. \\

%----------------------------------------------------------------------

\song{Evil Man}

\writtenby{Mackenzie/Smith}

It is 1864 now, and the American's war has not come to the \word{desert} lands. They fight among themselves way off to the north. The \word{Yavapai}-Apache are still lords of all they survey. \\

Then one morning the Americans did come led by a man holding a leather book with a cross stamped in the leather. An evil man who did terrible things to people in the name of a \word{God} that looked upon the man himself with revulsion. \\

\word[Miguel O'Brien]{Miguel} ran from his wickiup, half asleep when they attacked. A rifle butt sent him unconscious. When he came to, he was trussed-up, on his back, on the ground, looking up at the Americans. He had not been killed because they had noticed his blue eyes and knew he was one of them. \\

So at the age of 16, Miguel was back among his father's people. But once more a family he loved had been killed. This time by Americans. \\

%----------------------------------------------------------------------

\song{Fort Whipple}

\writtenby{Mackenzie/Smith}

\note{Chapter 3:}

The Americans took the trussed-up boy to a place called \word{Fort Whipple}, a fly-blown group of tents surrounded by a stone and timber stockade. \\

An American called Willis was the boss there and he glared at the man of \word{God} as he entered with his captives. He noticed the boy when he was brought in with a few \word{Yavapai} girls and he looked into the colour of his eyes. \\

``What do you make of him?'' he asked the God man. \\
``He may be the young O'Brien boy who was lost here years ago or he could be from the Jepson party that never made it to New Mexico,'' said the God man back. \\

They named the boy Jepson O'Brien but the natives and frontiersmen called him ``Blue'', because of his eyes but also because of the baleful, almost sad expression he carried on his face. The expression of someone who kills with compassion but not mercy. \\

Although he was still a boy, the men mostly kept away from him. All except for one, a trapper who understood his skills, and in return, fed him and taught him the white man's way. In a short while, he could speak, read, and write their language, and he also added the calm, fast dignity of the gunman to his arsenal. \\

He was so fast that men treated him with care. But he was slow to anger and when angry, swift and final in his reply. In the Arizona \word{desert} in the 1860's, he had every skill that you needed to survive. And he was just 17. \\

%----------------------------------------------------------------------

\song{The God Man's Goat Lust}

\writtenby{Mackenzie/Smith}

\note{Chapter 4:}

The \word{God} man with the \word{Bible} was in the back room of the chapel at \word{Fort Whipple.} The God man was deeply engrossed in satisfying his goat-lust with a \word{Yavapai} girl. She never said a damn thing but just leaned over an altar while he defiled her. He held a pistol to her head as he grunted away and when it was finished he shoved her towards the outside door. \\

But the God-man never got to fixing his long-johns or his black trousers. The young man named ``Blue'' strode softly up behind him and drove a long-bladed knife into his neck. Blood spurted into the chalice on the altar, but not the blood of the Christ. Just the blood of the God man. \\

With a cough he died. And a bubbled gurgle. \\

The young man named Blue took the Yavapai girl, money, guns and food, and two strong horses and rode into the \word{desert}, away from Fort Whipple. The God man's body was found but he was not missed. \\

%----------------------------------------------------------------------

\song{The Killing Ground}

\writtenby{Mackenzie/Smith}

\note{Chapter 5:}

For days they traveled, the young man and the \word{Yavapai} girl. She told him her name and they spoke in the language. They rode the horses until they gave out then their throats were slit and meat was taken to eat later. No fires were lit and they ate berries and raw jackrabbit as well to keep going. After a week they relaxed more as they entered Apacheria. \\

They saw dust way off like dust devils but they knew it was horses. They could hear shots and no more. When all was quiet a day later they moved silently towards the killing ground. \\

The buzzards told them the story before they got there. Dead white people, a lot of them, maybe a half dozen and burnt wagons and arrows. But not from one tribe. Some of the arrows were different and shod hoofmarks and moccasin tracks that were shaped like a white man's way of walking. Some white men had done this loosely disguised as Apache. \\

They took what they could use and walked on. The purple mountains and red ochre earth swallowed them up and the young man smelt his own blood as they ran. And it was a good smell. \\

The smell of being alive. \\

%----------------------------------------------------------------------

\song{Dust In the Wind}

\writtenby{Mackenzie/Smith}

Suddenly the girl pitched sideways and a split second later the young man heard the distant shot. He dived for some rocks and watched as more bullets hit the girl. The young man looked to her body and as she died he worked out where the shots were coming from. \\

And he knew that \word{death} was going to walk among the shooters. \\

\note{Chapter 6:}

There is one thing a white man should never do and that is move towards an Apache because you will never get there. How do you catch dust in the wind? The young man saw the way they were coming by the movement of insects and birds and he knew where to go. Like the \word{snake} he slithered into a dry arroyo and worked behind the shots in an arc. \\

After a while he saw them. Three men, three white men clad in skins and they walked confidently towards the girl. The young man knew somewhere behind them another one held the horses, making four all together. \\

He moved towards that man. The killers could wait. Let them enjoy the hunt before they went under. He found the one by the horses. He was young too and died quietly with a surprised indignant look on his face. The young man tied the horses to a tree. They'd come in handy later. Four horses and equipment. \\

At the girl's body, two men knelt beside her while another stood guard. The guard suddenly cried out as his head exploded in a bubble of pink spray and he fell forward. The other two went to ground and nervously called out to each other. \\

``Do you see the bastard?'' -- ``No, he must be close.'' \\

But he wasn't. A Sharp's sporting rifle will reach a long way in the right hands. The young man took careful aim and the smaller of the two men felt his right leg blasted away. The bigger, heavier man sank as far into the ground as he could make himself go. And still he could not see where the young man was. \\

%----------------------------------------------------------------------

\song[Guns \textamp{} Horses]{Guns \& Horses}

\writtenby{Mackenzie/Smith}

The young man by now was astride a horse and making for a \word{Yavapai} stronghold, half a day's ride away. He had more guns and horses than he needed and he knew where two white men were sitting in the \word{desert} with no water and no horses. White men dressed as Yavapai Apaches. \\

The white men would be calling for their mothers and their \word{God} by evening. The young man would be drunk on tiswin and full of deer meat. And satisfied by their agony. \\
